\chapter{Introduction} \label{chap:introduction}
Quantum computing has emerged as a transformative paradigm in computation, offering the potential to efficiently solve problems that are very hard to solve for classical computers \cite{gill2024quantum}. At the heart of quantum computation lies the concept of quantum circuits, which leverages quantum gates to manipulate qubits and perform computations. However, as quantum algorithms grow in complexity, so do their circuit representations. The increasing need for depth and size of quantum circuits introduce significant challenges, particularly in the presence of noise and decoherence in Noisy Intermediate-Scale Quantum (NISQ) devices \cite{Preskill_2018}. As a result, optimizing and compressing quantum circuits is a crucial step toward improving computational efficiency and the feasibility of real-world quantum applications.
\\[12pt]
One promising approach to circuit optimization is the utilization of higher-dimensional quantum systems, or qudits. Unlike conventional qubits, which operate with a two-level quantum state, qudits exploit high dimensional degrees of freedom, enabling more compact representations of quantum operations. By encoding information in multi-level systems, qudit-based architectures can reduce the number of required gates, particularly entangling gates, which are a major source of error in quantum computations. This approach has the potential to enhance quantum circuit efficiency and mitigate error propagation, contributing to the overall stability and performance of quantum algorithms \cite{gao2023role}.
\\[12pt]
This thesis explores the compression of quantum circuits using higher-dimensional systems. It investigates the mathematical foundations of quantum computation and the transformation of quantum circuits from qubit to qudit architectures. A program for quantum circuit compression is developed, leveraging graph-based clustering techniques to identify optimal qudit encodings. This program provides a flexible framework that can be extended to incorporate new optimization techniques and support further advancements in quantum circuit compression.
\\[12pt]
The structure of this thesis is as follows: Chapter \ref{chap:background} provides a theoretical background on quantum mechanics, qubits, and the role of entanglement. Chapter \ref{chap:circuits} discusses the construction and properties of quantum circuits, including universal gate sets. Chapter \ref{chap:compression} introduces the concept of quantum circuit compression using qudits, details the proposed algorithm, and describes its implementation. Finally, Chapter \ref{chap:conclusion} presents the conclusions and discusses possible extensions and improvements to the developed program, including support for additional gate sets, circuit optimization techniques, and packaging for seamless integration into other software.