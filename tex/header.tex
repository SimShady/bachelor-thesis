\usepackage{ifpdf}
\ifpdf
  \input{glyphtounicode.tex}    %Part of modern distribution
  \input{glyphtounicode-cmr.tex}     %Additional glyph: You must grab it from pdfx package
  \pdfgentounicode=1
  \pdfinterwordspaceon
  \usepackage[a-2u,pdf17]{pdfx}
  \pdfomitcharset 1
\else  %Place here the settings for other compilator
\fi
%Encoding + cmap (to get proper UTF8 mapping)
%------------------------------------------------------
\usepackage{cmap}
\usepackage[utf8]{inputenc} % Richtiges anzeigen von Umlauten und quasi allen anderen Schriftzeichen
\usepackage[T1]{fontenc} % Wichtig für alles was mehr als ASCII verwendet
\usepackage{csquotes} % Schöne Anführungsstriche mit \enquote{Text}
\usepackage{amsmath} % Bessere und schönere mathematische Formeln
\usepackage{mathtools} % Noch schönerere mathematische Formeln
\usepackage{bbm} % Identity matrix 1
\usepackage{amstext} % \text{} Macro in mathematischen Formeln
\usepackage{amsfonts} % Erweiterte Zeichensätze für mathematische Formeln
\usepackage{amssymb} % Spezielle mathematische Symbole.
\usepackage{amsthm}
\usepackage{tcolorbox}
%Correct UTF8 mapping for ams fonts
\usepackage{setspace} % for onehalfspacing and stuff
\ifdefined\pdffontattr% \ifdefined is part of the e-TeX extension, which is part of any modern LaTeX compiler.
    \immediate\pdfobj stream file {umsa.cmap}
    {\usefont{U}{msa}{m}{n}\pdffontattr\font{/ToUnicode \the\pdflastobj\space 0 R}}
    \immediate\pdfobj stream file {umsb.cmap}
    {\usefont{U}{msb}{m}{n}\pdffontattr\font{/ToUnicode \the\pdflastobj\space 0 R}}
\fi
\usepackage{array} % Matrizen in mathematischen Formeln
\usepackage{textcomp} % Für textmu und textohm etc. um im Fließtext keine Mathematik 
\usepackage{textalpha} % Damit können griechische Zeichen direkt im Text verwendet werden (siehe zeichen.txt)
\usepackage{paralist} % Für compactitem und compactenum
\usepackage{xstring} % Für IF in Titelseite
\usepackage{xfrac}

\usepackage[version=3]{mhchem} % Für Chemische Formeln
\usepackage{braket} % Für das quantenmechanische Bra-Ket
\usepackage{physics}

\usepackage{geometry} % Seitenränder und Seiteneigenschaften setzen
% \usepackage[showframe]{geometry} % Anzeigen der Seitenränder, nützlich für debugging. http://ctan.org/pkg/geometry

\usepackage[bottom]{footmisc} % Zwingt Fußnoten an das Ende der Seite
\usepackage[pdftex]{hyperref} % Links richtig anzeigen. Sowohl innerhalb des Dokuments (Fußzeilen, Formeln), als auch ins Internet

\usepackage[numbers, sort&compress]{natbib}
\bibliographystyle{naturemag}

\usepackage{graphicx} % Wichtig für das Einbinden von Grafiken
\usepackage{caption}
\usepackage{subcaption} % Einbinden von mehreren Grafiken in einer figure
\usepackage{float}

\usepackage{dirtree} % Erlaubt das erstellen von Dateibäumen
% \dirtreecomment{Text} erstellt einen Kommentar zu dem Verzeichnis bzw. der Datei
\newcommand{\dirtreecomment}[1]{\dotfill{} \begin{minipage}[t]{0.5\textwidth}#1\end{minipage}}

\usepackage{fancyvrb} % Mehr Optionen für Verbatim
\usepackage{listings} % Zur Darstellung von Programmcode
\usepackage{pdflscape} % Querformat Seiten
\usepackage{fancyhdr}

\newcommand{\writeIn}[1]{\usepackage[#1]{babel}} % Definiert einen neuen Befehl um die Sprache des Dokuments zu setzen

\usepackage{colorprofiles}
\PassOptionsToPackage{usenames,dvipsnames}{color}
\usepackage{color} % Farben für den todo Befehl
\newcommand{\todo}[1]{{\color{olive}(TODO: #1)}} % Einfach \todo{Text} verwenden!

\newcommand{\blankpage}{ \newpage \thispagestyle{empty} \mbox{} \newpage }

\fancypagestyle{myheader}{
    \fancyhf{}
    \fancyhead[EL]{\thepage}
    \fancyhead[ER]{\leftmark}
    \fancyhead[OL]{\leftmark}
    \fancyhead[OR]{\thepage}
    \renewcommand{\headrulewidth}{0.4pt}
}
\renewcommand{\chaptermark}[1]{\uppercase{\markboth{#1}{}}}
\renewcommand{\chaptermarkformat}{} % Entferne das Wort "Chapter"

\fancypagestyle{plain_center_pagenumber}{
	\fancyhf{}
    \fancyfoot[C]{\thepage}
    \renewcommand{\headrulewidth}{0pt}
}

\RedeclareSectionCommand[
  beforeskip=-1em, % entfernt den Abstand über der Überschrift
  afterskip=1em, % fügt einen Abstand unter der Überschrift ein
  font=\normalfont\Large\bfseries % ändert die Schriftgröße und den Stil
]{chapter}

\RedeclareSectionCommand[
  beforeskip=-1em, % entfernt den Abstand über der Überschrift
  afterskip=1em, % fügt einen Abstand unter der Überschrift ein
  font=\normalfont\bfseries % ändert die Schriftgröße und den Stil
]{section}

\RedeclareSectionCommand[
  beforeskip=-1em, % entfernt den Abstand über der Überschrift
  afterskip=1em, % fügt einen Abstand unter der Überschrift ein
  font=\normalfont\small\bfseries % ändert die Schriftgröße und den Stil
]{subsection}

\newcommand{\Chi}{\raisebox{2pt}{$\chi$}}

\definecolor{codegreen}{rgb}{0,0.6,0}
\definecolor{codegray}{rgb}{0.5,0.5,0.5}
\definecolor{codepurple}{rgb}{0.58,0,0.82}
\definecolor{backcolour}{rgb}{0.95,0.95,0.92}
\definecolor{stringColor}{HTML}{718a62}
\lstdefinelanguage{Nix}{
	keywords=[1]{true, false, null},
	keywords=[2]{let, in, with, rec, inherit},
	keywords=[3]{toString},
	keywordstyle=\color{blue},
	sensitive=true,
	% comments
	comment=[l]{\#},
	morecomment=[s]{/*}{*/},
	commentstyle=\color{gray},
	% strings
        morestring=*[d]{"},
        morestring=[s][\color{red}]{\$\{}{\}},
        morestring=*[d]{''},
	stringstyle=\color{stringColor},
}

\lstdefinestyle{codestyle}{
  backgroundcolor=\color{backcolour},   
  commentstyle=\color{codegreen},
  keywordstyle=\color{magenta},
  numberstyle=\tiny\color{codegray},
  stringstyle=\color{codepurple},
  basicstyle=\ttfamily\footnotesize,
  breakatwhitespace=false,         
  breaklines=true,                 
  captionpos=b,                    
  keepspaces=true,                 
  numbers=left,                    
  numbersep=5pt,                  
  showspaces=false,                
  showstringspaces=false,
  showtabs=false,                  
  tabsize=2
}

\lstset{style=codestyle}

\newtheorem{postulate}{Postulate}
\newtcolorbox{postulatebox}{colback=white, colframe=white, left=10mm, right=10mm, boxrule=0pt}
\newtheorem{definition}{Definition}
\newtcolorbox{definitionbox}{colback=white, colframe=white, left=8mm, right=8mm, boxrule=0pt}
\setlength{\parindent}{0pt} % No indentation