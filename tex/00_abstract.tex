\chapter*{\abstractname}
\addcontentsline{toc}{chapter}{\abstractname}
Quantum computing has the potential to solve complex problems beyond the reach of classical computers, but the need for implementing quantum circuits with increasing depth and size pose challenges in terms of noise, decoherence, and hardware limitations. One approach to mitigate these issues is quantum circuit compression, which aims to reduce the number of required quantum gates, particularly entangling gates, that contribute to error propagation. This thesis explores the use of higher-dimensional quantum systems, known as qudits, to achieve more compact and efficient quantum circuit representations. By encoding quantum information in multi-level systems, qudits allow for a reduction in circuit depth and the number of entangling operations.
\\[12pt]
This work investigates the mathematical foundations of quantum circuit transformation from qubit-based to qudit-based architectures and presents an algorithm for circuit compression based on graph-based clustering techniques. A software implementation of the algorithm is developed to automate the conversion process, demonstrating the feasibility of qudit-based optimization. The results indicate that qudit-based circuits can lead to improved efficiency and lower error rates, offering a promising path toward enhancing quantum computation on Noisy Intermediate-Scale Quantum (NISQ) devices. Future work includes refining optimization techniques, expanding support for diverse quantum gate sets, and integrating the developed framework with existing quantum computing platforms.
\newpage