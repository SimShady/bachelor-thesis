\chapter{Conclusion} \label{chap:conclusion}
This thesis explored the concept of quantum circuit compression using higher-dimensional quantum systems (qudits) as an approach to optimizing quantum computations. By leveraging qudits, it is possible to reduce circuit width and the number of entangling gates, which are known to be a major source of noise and decoherence in Noisy Intermediate-Scale Quantum (NISQ) devices. The research focused on the transformation of qubit-based circuits into qudit-based representations and implemented a framework to compress qubit circuits to qudit circuits using clustering algorithms to identify optimal qudit encodings.
\\[12pt]
The results demonstrate that qudit-based architectures can effectively reduce the number of entangling operations while maintaining computational integrity. The proposed algorithm provides a flexible framework for circuit compression and can be extended to support additional optimization techniques. The developed program successfully converts quantum circuits into qudit equivalents, theoretically enabling more efficient execution on quantum hardware that supports higher-dimensional states.
\\[12pt]
Future enhancements to the program could expand support for a broader range of universal gate sets, enabling applications on real quantum hardware. Additionally, refining existing clustering algorithms to incorporate machine learning techniques may further enhance the optimization process. To simplify its adoption in practical use cases, the program could be packaged as a modular software library with well-defined APIs, allowing seamless integration into custom circuit compression applications.
\\[12pt]
By providing a framework for quantum circuit compression, this work contributes to the ongoing effort to improve quantum computing efficiency, paving the way for more scalable and noise-resilient quantum algorithms.